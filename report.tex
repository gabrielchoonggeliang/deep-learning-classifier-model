\documentclass{article}

\usepackage{hyperref}

\title{Automotive Brand Recognition using ResNet18}
\author{Gabriel Choong Ge Liang, Lim Yong Yang, Thong Meng Hin}
\date{\today}

\begin{document}

\maketitle

\section{Introduction}
In this rapidly evolving modern era, technology continues to advance at an unprecedented pace, with Deep Learning emerging as a potent tool for image recognition and classification. Our primary objective in this project is to develop a model capable of recognizing automotive brands, leveraging the robust capabilities of ResNet18.

The burgeoning automotive industry in our country underscores the importance of this project. An envisioned system capable of recognizing car brands could significantly impact market research, streamline traffic management, and provide precise user recommendations efficiently.

The selection of Proton, Perodua, Toyota, and Honda as primary input data is rooted in their prevalence within Malaysia, commonly seen on our roads. This choice aims for accuracy within our project scope. Expanding the dataset to encompass all car brands could amplify its size and scope.

We employ the ResNet-18 architecture, a convolutional neural network, to process our dataset consisting of nine hundred car photos. The project's culmination seeks to demonstrate ResNet's effectiveness in distinguishing automotive brands, enhancing computer vision applications in the automotive industry.

\section{Literature Review}
In the rapidly evolving landscape of technology, Deep Learning has emerged as a powerful tool for image recognition and classification. This literature review investigates existing research related to Deep Learning and ResNet applications.

Deep Learning, albeit demanding more training time and computing resources than traditional Machine Learning, offers superior prediction accuracy and generalization ability. It excels in automatically and efficiently extracting features from images, particularly differentiating images with subtle nuances that traditional methods struggle to recognize. This strength makes Deep Learning highly suitable for image recognition tasks. Models trained using Deep Learning algorithms have achieved remarkable results in large-scale classification tasks in Computer Vision. This approach, despite its challenges, illuminates Neural Networks and deep architectures, prompting active research exploration.

Our project's data is processed using ResNet, a Convolutional Neural Network that focuses on exploiting specific input knowledge. While recent years have seen a slowdown in Neural Network research for image analysis, it's important to dispel the misconceptions surrounding the complexity of modeling these powerful Machine Learning algorithms.

ResNet, also known as a Residual Neural Network, employs weight layers to learn residual functions related to the layer inputs. Its unique structure facilitates the training of deeper networks by leveraging skip connections performing identity mappings. Despite challenges encountered in training deeper networks, ResNet's approach holds promise, addressing the "degradation" problem and pushing the boundaries of deep architectures.

In summary, this literature review underscores the potential of Deep Learning and ResNet in image recognition. Building upon these findings, our research aims to further advance computer vision applications in the automotive industry.

\section{Methodology}
\subsection{Data Preprocessing}

Our initial dataset organization focused on optimal training and validation. This meticulous process included defining transformation and normalization procedures crucial for effective model learning. The data\_transforms dictionary facilitated distinct data splits for training and validation. Segregation of the dataset into these subsets was accomplished using the random\_split function, followed by the implementation of data loaders for streamlined handling of both sets.

\subsection{Model Architecture}

The model training loop commenced with the setup of training history lists for loss and accuracy. This established a foundation for recording and analyzing performance metrics during the training process. Visualization of training data was achieved through grid representation using torchvision.utils.make\_grid.

\subsection{Model Training}

Our training routine employed cross-entropy loss to evaluate how accurately predicted probabilities aligned with actual class labels. This method is particularly effective for image classification tasks.

Stochastic Gradient Descent (SGD) served as our optimizer during training. In contrast to regular gradient descent, SGD introduces randomness, often yielding better real-world results. Employing a learning rate scheduler ensured a balanced approach to optimizing the model.

Critical hyperparameters, including learning rate, momentum, step size, gamma, and the number of epochs, were established. The ResNet18 model was initialized, and its last fully connected layer was adapted to accommodate the dataset's class count.

\subsection{Transfer Learning}

Transitioning to transfer learning, a pre-trained ResNet18 model was employed, with adjustments made to the final fully connected layer to suit our custom dataset. The focus during training remained on optimizing only the final layer's parameters, leveraging knowledge acquired from the pre-trained model.

\subsection{Training History}

Matplotlib played a pivotal role in visualizing our training and validation process. Plots illustrating the progression of loss and accuracy over epochs provided valuable insights into the model's learning dynamics.

\section{Experimental Results}
\subsection{Training and Validation Loss}

Experimental results revealed relatively high training and validation loss in our model. This can be attributed to noisy data within the dataset, impacting the model's ability to accurately classify images. Additionally, the achieved accuracy does not meet the desired standards.

Despite these limitations, utilizing ResNet18 for classification tasks exhibits promise in image recognition. The model's ability to discern complex features and differentiate automotive brands remains a notable strength.

\section{Conclusion}
This project centered on developing a ResNet18-based classification model for automotive brand recognition. Despite challenges such as noisy data and suboptimal accuracy, the utilization of ResNet18 showcased potential in image recognition and brand differentiation.

Enhancing the model's performance requires addressing noisy data through effective data cleaning or acquiring a more reliable dataset. Further optimization through fine-tuning hyperparameters and exploring alternative preprocessing methods could yield improved accuracy.

Future research directions could involve exploring alternative deep learning architectures or employing ensemble methods to enhance classification performance.

In summary, this project underscores deep learning's potential in the automotive industry and highlights ResNet18's effectiveness in image recognition tasks. With continued refinement and exploration, this model stands to contribute significantly to applications like market research, traffic management, and user recommendation systems.

\section{Data and Source Code}
For access to the data and source code utilized in this project, please refer to the following link: 


\href{https://drive.google.com/drive/folders/1S3XY0bmViWT4PWdOAeBQNUHMYYkH8b9U?usp=sharing}{Google Drive Link}


\end{document}
